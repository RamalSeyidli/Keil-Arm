\documentclass[pdftex,12pt,a4paper]{article}

\usepackage{graphicx}  
\usepackage[margin=2.5cm]{geometry}
\usepackage{breakcites}
\usepackage{indentfirst}
\usepackage{pgfgantt}
\usepackage{pdflscape}
\usepackage{float}
\usepackage{epsfig}
\usepackage{epstopdf}
\usepackage[cmex10]{amsmath}
\usepackage{stfloats}
\usepackage{multirow}
\usepackage{listings}
\usepackage{minted}

\lstdefinestyle{customasm}{
  belowcaptionskip=1\baselineskip,
  frame=L,
  xleftmargin=\parindent,
  language=[x86masm]Assembler,
  basicstyle=\footnotesize\ttfamily,
  commentstyle=\itshape\color{purple!40!black},
  morekeywords={LDR, BEQ, B, SUBS, BLT, LSLS, ORRS, LSRS, ADDS, BNE, MULS, ANDS}
}

\lstset{
    language=[ARM]Assembler
    basicstyle=\ttfamily\small,
    aboveskip={1.0\baselineskip},
    belowskip={1.0\baselineskip},
    columns=fixed,
    extendedchars=true,
    breaklines=true,
    tabsize=4,
    prebreak=\raisebox{0ex}[0ex][0ex]{\ensuremath{\hookleftarrow}},
    frame=lines,
    showtabs=false,
    showspaces=false,
    showstringspaces=false,
    keywordstyle=\color[rgb]{0.627,0.126,0.941},
    commentstyle=\color[rgb]{0.133,0.545,0.133},
    stringstyle=\color[rgb]{01,0,0},
    numbers=left,
    numberstyle=\small,
    stepnumber=1,
    numbersep=10pt,
    captionpos=t,
    escapeinside={\%*}{*)}
}

\renewcommand{\refname}{REFERENCES}
\linespread{1.3}

\usepackage{mathtools}
%\newcommand{\HRule}{\rule{\linewidth}{0.5mm}}
\thispagestyle{empty}
\begin{document}
\begin{titlepage}
\begin{center}
\textbf{}\\
\textbf{\Large{ISTANBUL TECHNICAL UNIVERSITY}}\\
\vspace{0.5cm}
\textbf{\Large{COMPUTER ENGINEERING DEPARTMENT}}\\
\vspace{2cm}
\textbf{\Large{BLG 212E\\ MICROPROCESSOR SYSTEMS\\ TERM PROJECT}}\\
\vspace{2.8cm}
\begin{table}[ht]
\centering
\Large{
\begin{tabular}{lcl}
\textbf{DATE}  & : & 31.01.2021 \\
\textbf{GROUP NO}  & : & G32 \\
\end{tabular}}
\end{table}
\vspace{1cm}
\textbf{\Large{GROUP MEMBERS:}}\\
\begin{table}[ht]
\centering
\Large{
\begin{tabular}{rcl}
150180116  & : & Ömür Fatmanur Erzurumluoğlu \\
150180053  & : & Mehmet Karaaslan \\
150180734  & : & Sinan Şar \\
150180901  & : & Ramal Seyidli \\
150170016  & : & Ümit Başak \\
\end{tabular}}
\end{table}
\vspace{2.8cm}
\textbf{\Large{FALL 2020}}

\end{center}

\end{titlepage}

\thispagestyle{empty}
\addtocontents{toc}{\contentsline {section}{\numberline {}FRONT COVER}{}}
\addtocontents{toc}{\contentsline {section}{\numberline {}CONTENTS}{}}
\setcounter{tocdepth}{4}
\tableofcontents
\clearpage

\setcounter{page}{1}

\section{INTRODUCTION}

In this project we were required to create a sorted set linked list structure using ARM assembly language. The program has to
read the data and operation flags from the input data set arrays and perform an operation in each System Tick ISR. The System Tick Timer will stop if the program reads all the data in the input datasets.

\section{MATERIALS AND METHODS}
All the functions and methods used in our implementation are explained in a deeply detailed fashion in the subparts below.

\subsection{Clear\_Alloc()}
\par This function changes all bits of the allocation table to 0.
\begin{lstlisting}[caption=Clear\_Alloc, style=customasm]
Clear_Alloc		FUNCTION			
;//-------- <<< USER CODE BEGIN Clear Allocation Table Function >>> ----------------------																			
				LDR 	R2, =AT_MEM			;loads start address of allocation table to r2
				LDR 	R0, =AT_SIZE		;loads size of allocation table to r0
				MOVS	R1, #0				;loads 0 to r1
Clear_loop		SUBS	R0, R0, #4			;decreases the index by 4 to clear from last element of allocation table to first element
				STR		R1, [R2,R0]			;assigns 0
				CMP		R0, R1				;compares 0 and R0 (index)
				BNE		Clear_loop			;if the first element is not reached, branches to Clear_loop
				BX 		LR					;branches to main				
;//-------- <<< USER CODE END Clear Allocation Table Function >>> ------------------------				
				ENDFUNC
\end{lstlisting}

\subsection{Clear\_ErrorLogs()}
\par This function clears all cells in the Error Log Array in case the memory addresses contain unwanted values.
\begin{lstlisting}[caption=Clear\_ErrorLogs, style=customasm]
Clear_ErrorLogs	FUNCTION			
;//-------- <<< USER CODE BEGIN Clear Error Logs Function >>> ----------------------																		
					LDR 	R2, =LOG_MEM			;loads start address of error log array to r2
					LDR 	R0, =LOG_ARRAY_SIZE		;loads size of error log array to r0
					MOVS	R1, #0					;loads 0 to r1
Clear_error_loop	SUBS	R0, R0, #4				;decreases the index by 4 to clear from last element of error log array to first element
					STR		R1, [R2,R0]				;assigns 0
					CMP		R0, R1					;compares 0 and R0 (index)
					BNE		Clear_error_loop		;if the first element is not reached, branches to Clear_error_loop
					BX 		LR						;branches to main				
;//-------- <<< USER CODE END Clear Error Logs Function >>> ------------------------				
				ENDFUNC
\end{lstlisting}

\newpage

\subsection{Init\_GlobVars()}
\par This function sets the global variables that are used in the program which are TICK\_COUNT, FIRST\_ELEMENT, INDEX\_INPUT\_DS, INDEX\_ERROR\_LOG and PROGRAM\_STATUS to 0. 
\begin{lstlisting}[caption=Init\_GlobVars, style=customasm]
Init_GlobVars	FUNCTION			
;//-------- <<< USER CODE BEGIN Initialize Global Variables >>> ----------------------		
				MOVS	R1, #0						;assigns 0 to R1
				LDR 	R0, =TICK_COUNT				;loads the address of TICK_COUNT global value
				STR		R1, [R0]					;assigns 0 to TICK_COUNT
				LDR 	R0, =FIRST_ELEMENT			;loads the address of FIRST_ELEMENT global value
				STR		R1, [R0]					;assigns 0 to FIRST_ELEMENT
				LDR 	R0, =INDEX_INPUT_DS			;loads the address of INDEX_INPUT_DS global value
				STR		R1, [R0]					;assigns 0 to INDEX_INPUT_DS
				LDR 	R0, =INDEX_ERROR_LOG		;loads the address of INDEX_ERROR_LOG global value
				STR		R1, [R0]					;assigns 0 to INDEX_ERROR_LOG
				LDR 	R0, =PROGRAM_STATUS			;loads the address of PROGRAM_STATUS global value
				STR		R1, [R0]					;assigns 0 to PROGRAM_STATUS (Program started)
				BX		LR							;branches to main			
;//-------- <<< USER CODE END Initialize Global Variables >>> ------------------------				
				ENDFUNC
\end{lstlisting}

\subsection{SysTick\_Init()}

In the SysTick\_Init function, the SysTick Registers are set to interrupt the program in the determined period. As stated in the assignment, period of the System Tick Timer interrupt is 964 us and CPU clock frequency is 16 MHz. The reload value is calculated as 15423 by using equation (1). The reload value is loaded to the SysTick Reload Value Register. The SysTick Current Value Register that counts down from the reload value to zero is set to 0. The SysTick Control and Status Register is assigned as 7 in order to set CLKSOURCE, TICKINT and ENABLE as 1. Since the timer will start after this function, the value of PROGRAM\_STATUS becomes 1. 

\begin{equation}
    Period~Of~the~System~Tick~Timer~Interrupt = (Reload~Value + 1)/Clock~Frequency
\end{equation}


\begin{lstlisting}[caption=SysTick\_Init Function, style=customasm]
SysTick_Init	FUNCTION			
;//-------- <<< USER CODE BEGIN System Tick Timer Initialize >>> ----------------------	
                LDR	 R0, =0xE000E014			;loads the address of SysTick Reload Value Register
				LDR	 R1, =15423					;assigns computed reload value (15423) to R1
				STR	 R1,[R0]					;assigns 15423 to SysTick Reload Value Register

				LDR  R0, =0xE000E018			;loads the address of SysTick Current Value Register
				MOVS R1, #0						;assigns 0 to R1
				STR  R1,[R0]					;assigns 0 to SysTick Current Value Register
				
				LDR	 R0, =0xE000E010			;loads the address of SysTick Control and Status Register
				MOVS R1, #7						;assigns 7 to R1
				STR	 R1, [R0]					;assigns 1 to CLKSOURCE, TICKINT and ENABLE
				
				LDR	 R0, =PROGRAM_STATUS		;loads the address of PROGRAM_STATUS global value
				MOVS R1, #1						;assigns 1 to R1
				STR	 R1, [R0]					;assigns 1 to PROGRAM_STATUS (Timer started)
				
				BX	 LR							;branches to main
;//-------- <<< USER CODE END System Tick Timer Initialize >>> ------------------------				
				ENDFUNC
\end{lstlisting}

\subsection{SysTick\_Handler()}

In the SysTick\_Handler function, the input data and its flag are read from IN\_DATA and IN\_DATA\_FLAG areas. When the flag value is 0, the input value is found in the linked list and removed from the list. If the linked list is empty, error code becomes 3, and if the input data cannot be found in the list, it becomes 4 during remove operation. When the flag value is 1, Insert function is called to insert the input value to linked list. If no allocable area is found in the insert operation, the error code is set to 1 and if the input is an existing value in the linked list, error code becomes 2. When flag is 2, the linked list is transformed to the array. Error code becomes 5 if the linked list is empty and the error code becomes 6 for other flag values. If the operation is performed properly, the error code is set to success code. 

\begin{lstlisting}[caption=SysTick\_Handler Function, style=customasm]
SysTick_Handler	FUNCTION			
;//-------- <<< USER CODE BEGIN System Tick Handler >>> ----------------------																	
				EXPORT SysTick_Handler
					
				PUSH	{LR}					;pushes LR to stack
				
				LDR 	R7, =TICK_COUNT			;loads address of TICK_COUNT to r7
				LDR		R1, [R7]				;loads the value of TICK_COUNT to r1
				MOVS	R6, #4					;assigns 4 to R6
				MULS	R6, R1, R6				;multiplies TICK_COUNT value and 4
				LDR		R5, =IN_DATA_FLAG		;loads start address of flags of input dataset to r5
				LDR		R2, [R5, R6]			;R2 stores the flag (operation)
				LDR		R5, =IN_DATA			;loads start address of input dataset to r5
				LDR		R3, [R5, R6]			;R3 stores the input data
				MOVS	R0, R3					;R0 also stores the input data
				PUSH	{R7,R1,R2,R3}			;pushes the registers to stack 
				
				CMP		R2, #0					;if the flag is 0
				BEQ		remove					;branches to remove
				CMP		R2, #1					;if operation flag is 1, call insert.
				BEQ		call_insert				;branches to call_insert
				CMP		R2, #2					;if operation flag is 2
				BEQ		transform				;branches to transform
				BNE		error					;if not equal, branches to error
call_insert		BL		Insert					;branches with link to Insert function
				B		pp						;after insert operation, branches to pp
remove			BL		Remove					;branches with link to Remove function
				B		pp						;after remove operation, branches to pp
transform		BL		LinkedList2Arr			;branches with link to LinkedList2Arr function
				B		pp						;after transform operation, branches to pp
				
error			MOVS	R0, #6					;R0 stores the error code as 6
\end{lstlisting}

After the operation, WriteErrorLog function is called to write the index, error code, flag and input data in LOG\_MEM area. At the end of this function, it checks that all data is read. If all data is read, the SysTick\_Stop function is called and the PROGRAM\_STATUS is set to 2.

\begin{lstlisting}[caption=SysTick\_Handler Function, style=customasm]
pp				POP		{R7,R1,R2,R3}			;pops from stack to the registers
				PUSH	{R7,R1}					;pushes the registers to stack 
				MOVS	R7, R1					;to interchange R1 and R0
				MOVS	R1, R0					;R1 stores the error code
				MOVS	R0, R7					;R0 stores the index
				BL		WriteErrorLog			;branches with link to WriteErrorLog function
				POP		{R7,R1}					;pops from stack to the registers
				ADDS	R1, #1					;increases R1 value by 1
				STR		R1, [R7]				;loads R1 to TICK_COUNT
				
				LDR		R5, =IN_DATA_FLAG		;loads start address of flags of input dataset to r5
				LDR		R3, =END_IN_DATA_FLAG 	;loads end address of flags of input dataset to R3
				MOVS	R6, #4					;assigns 4 to R6
				MULS	R6, R1, R6				;multiplies TICK_COUNT value and 4
				ADDS	R6, R5, R6				;sums the start adress of flags array and TICK_COUNT		
				CMP		R3, R6					;compares the sum and the end adress of the array
				BEQ		SysTick_Stop			;if equal, branches to SysTick_Stop		
				LDR 	R0, =PROGRAM_STATUS		;Load Program Status Variable Addresses.
				POP		{PC}					;returns to where the SysTick_Handler function was called				
				
;//-------- <<< USER CODE END System Tick Handler >>> ------------------------				
				ENDFUNC
\end{lstlisting}

\subsection{Insert(value)}
\par The Insert function takes the input value that is coming from systick handler function with the R0 register then, iterates the linked list, compares the values that are in the linked list and inserts the new value to the correct place. While iterating, if a duplicate value is encountered, the function returns to the systick handler with an error code. There are four major cases that the program may encounter while the insertion:
\begin{lstlisting}[caption=Insert Function, style=customasm]
Insert			FUNCTION			
;//-------- <<< USER CODE BEGIN Insert Function >>> ----------------------					
				PUSH	{LR}				;Push LR, will pop pc in order to return to systick_handler
				LDR		r5,=FIRST_ELEMENT	;store head pointer at r5
				LDR 	r1,=DATA_MEM		;store value of head at r1
				LDR		r1,[r1]		;load r1 with first nodes data		
				LDR		r2,[r5]		;load r2 with heads address
				;LDR		r2,[r2]
			
			;THis should be done for R0 in handler ;;LDR		r2,=[IN_DATA,TICK_COUNT]	;;!!!!!input == 0 and list is empty, there will be some issues..
				
				CMP		r2,#0				;Check if list is empty
				BEQ		Empty_list			;if it is, branch to empty list case
				
				CMP 	r0,r1				;Check if input is smaller than head
				BLT		New_Head		;if it is, branch to add new head
				MOVS	r1,r5				;move head pointer to r1
				LDR		r1,[r1]			;load r1 with heads address
				B		Check			;branch to case checking loop
\end{lstlisting}
\newpage
\begin{itemize}
    \item Inserting to an empty list: While inserting to an empty list, first, the function puts the input to the first space in the DATA\_MEM and then changes the last bit of the Allocation Table as 1 and returns to the systick handler with a success code.
    \begin{lstlisting}[caption=Inserting to an empty list, style=customasm]
    Empty_list		LDR		r1,=DATA_MEM		;set r1 as Data_MEM's start address
				STR		r1,[r5]
				STR		r0,[r1]				;Set head of the list as input		
				LDR		r4,=(__AT_END-4)	;load end address of allocation table to r4
				LDR		r5,=0x00000001		;prepare value to store in allocation table.(32 bits 000...001)
				STR 	r5, [r4]			;Set allocation table 
				MOVS	r5,#1				;Load 1 to r5
				MOVS	r0,#0				;Set r0 with success code
				;LDR		r2,=IS_EMPTY		;load  IS_EMPTY 's address to r2
				;STR		r5,[r2]				;Set IS_EMPTY variable as 1, since list is not empty anymore
				
				;;This doesn't require malloc, we don't think...
				POP		{PC}				;return to systick_handler
    \end{lstlisting}
    \item Inserting to the head of the list: While inserting to the head of the list, first, the function calls the malloc function, the malloc function returns an appropriate address from the DATA\_MEM and changes the Allocation Table's corresponding bit to 1. Then, the insert function stores the new input value to the DATA\_MEM and changes the DATA\_MEM's second word as previous head's address. Finally it makes the new inserted element the FIRST\_ELEMENT. While this function is working, if the malloc function returns an error, it branches to Mal\_Ins\_Error and assigns 1 to R0 to indicate that a malloc error occured, and returns to systick handler.
    \newpage
    \begin{lstlisting}[caption=Inserting before the head of the list, style=customasm]
    New_Head		PUSH	{r0,r1,r5}			;push values before calling malloc
				BL		Malloc				;call malloc
				MOVS	r2,r0				;Save address coming from malloc to R2
				POP		{r5,r1,r0}			;pop
				CMP		r2,#0				;if malloc raised an error
				BEQ		Mal_Ins_Error			;branch to error label
				STR		r0,[r2]				;store input value at allocated memory address
				LDR		r1,[r5]				;load current heads address to r1
				;LDR	r1,[r1]
				STR		r1,[r2,#4]			;store next pointer at allocated memory address.
				STR 	r2, [r5]			;make the new head first_element
				MOVS	r0,#0				;Set r0 with success code
				POP		{PC}				;return to systick_handler
    \end{lstlisting}
    \item Inserting in between the nodes: While inserting in between the nodes, similar to the previous case, the function calls the malloc function, stores the input in the first word of DATA\_MEM, it changes the next elements address as next address for the new element and changes new elements address as current elements next address. Similar to the previous case, it returns to systick handler if an error occurs.
    \begin{lstlisting}[caption=Inserting in between the list, style=customasm]
    Check			
				MOVS	r3,r1				;Load r3 with current nodes next address
				LDR		r6,[r3,#4]			;Lad r6 with next nodes next value
				LDR		r4,[r3]				;load r4 with next nodes data
				CMP		r0,r4				;check if input is duplicate
				BEQ		Duplicate_Error			;branch to error table
				CMP		r0,r4				;compare input with next node,if its lesser then we found where to insert
				BLT		found_insert			;branch to insert new node
				CMP		r6,#0				;Check if we are at the tail
				BEQ		add_tail			;if we are, branch to add new tail
				;MOVS	r1,r4				;else, load r1 with next nodes data
				MOVS	r7,r1				;load r1 with current nodes data
				MOVS	r1,r6				;else, load r1 with next nodes next
				B		Check		

found_insert	PUSH	{r0,r1,r3,r5,r7}	;push values before calling malloc
				BL		Malloc				;call malloc
				MOVS	r2,r0				;Save address coming from malloc to R2
				POP		{r7,r5,r3,r1,r0}	;pop
				CMP		r2,#0				;if malloc raised an error
				BEQ		Mal_Ins_Error			;branch to error label
				STR		r0,[r2]				;store input value at allocated memory address
				STR		r3,[r2,#4]			;store next elements data address as next for new element
				STR		r2,[r7,#4]			;store new elements data address as current elements next.
				MOVS	r0,#0				;Set r0 with success code
				POP		{PC}				;return to systick_handler
    \end{lstlisting}
    \newpage
    \item Inserting after the last element: While inserting after the last element, similar to the previous cases, the function calls the malloc function, stores the input in the first word of DATA\_MEM, changes the old tails next address as the new elements address and sets new elements next address as null. Similar to the previous case, it returns to systick handler if an error occurs in malloc.
    \begin{lstlisting}[caption=Inserting after the last element of the list, style=customasm]
    add_tail		PUSH	{r0,r1,r3,r4}		;push values before calling malloc
				BL		Malloc				;call malloc
				MOVS	r2,r0				;Save address coming from malloc to R2
				POP		{r4,r3,r1,r0}		;pop
				CMP		r2,#0				;if malloc raised an error
				BEQ		Mal_Ins_Error			;branch to error label
				STR		r0,[r2]				;store input value at allocated memory address	
				STR		r2,[r3,#4]			;Load old tails next with new tails data address
				MOVS	r4,#0				;Load r4 with next nodes data
				STR		r4,[r2,#4]			;Set new tails next as null.
				MOVS	r0,#0				;Set r0 with success code
				POP		{PC}
    \end{lstlisting}
\end{itemize}
\newpage
\subsection{Malloc()}
The Malloc function takes the input value that is coming from Insert function with  the  Register 0  then,  iterates  the allocation table to check for empty space in the allocation table. If there is space for a new node, a memory address is returned to Insert function correspondinglly. If there is not any empty space in the allocation table, Malloc will return an error code to the Insert function via Register 0.
\begin{itemize}
    \item Allocation Table: Malloc function finds the free memory spaces and allocates them using the allocation table.
The allocation table consists of 20 words and each word has 32 bits. Each bit corresponds to one memory node. If the bit is 0, the memory node is free and a new input can be allocated there. We used the allocation table like the figure present in the assignment pdf. Nodes are allocated in each row starting from the least significant bit of the 32 bits. We also selected the last word of the allocation table memory area to be the first row, just to make it easier to visualise. The whole table is allocated from right to left visually.
\item Connection between Allocation Table and Data Memory: Each bit in allocation table refers to a node in the Data Memory area. This relation is set in a very straightforward manner. The first bit in the Allocation Table is related with the first 2 words (1 word for data, 1 word for next pointer) of the Data Memory. All other bits and the space is similarly related to each other.
\item Iteration: Malloc function starts iterating on the first row of the allocation table. This iteration is done by a loop, throughout the loop each bit is checked for their value with a mask. If a free space is found, the program branches to the found label. If not, the word gets shifted and masked again to check the next bit. If all bits in the row are checked and no free space is found the program branches to the newline label, which advances to a new row in the allocation table.

\begin{lstlisting}[caption=Malloc iteration, style=customasm]
Malloc			FUNCTION			
;//-------- <<< USER CODE BEGIN System Tick Handler >>> ----------------------	
				LDR 	r1,=DATA_MEM		;load start address of linked list to r1
				LDR 	r2,=AT_MEM			;load start address of allocation table to r2
				LDR		r4,=(__AT_END-4)	;load end address of allocation table to r4
				LDR		r5,=0xFFFFFFFE		;mask for each line on allocation table.(32 bits 111...110)
				MOVS	r7,#0				;keep an index for iterations at r7
				LDR		r6,[r4]				;load 32 bits of allocation table to r6. The 32 bits will change as we iterate on the table.
iterate_mal		CMP		r7,#32				;check if we looked at all 32 bits in allocation table line, if true go to next line
				BEQ		new_line
				PUSH	{r7}				;push r7 to preserve its value
				MOVS	r7,r6				;move r6 to r7
				ORRS	r7,r5,r7			;mask r6 value to check if lsb is 0					
				CMP 	r7,r5				;if it is 0, we found a free memory space for a new node.
				BEQ		found
				POP		{r7}				;if not, pop r6 and keep iterating on the allocation table line.
				LSRS	r6,#1				;shift r6 value to right to check the next bit
				ADDS	r7,#1				;increment loop index by one
				ADDS	r1,#8				;increment linked list node address by 8.(4 for data,4 for next)
				B		iterate_mal			;return to start of the loop	
\end{lstlisting}

\item   Newline label: This label advances to the next row on the allocation table. If the whole table has been checked and no free space is found, it branches to the error label.

\begin{lstlisting}[caption=Malloc newline label, style=customasm]
new_line		SUBS	r4,#4				;iterate to a new line in the allocation table
				CMP		r4,r2				;if we checked the whole table and there is no empty space
				BLT		mal_error			;go to error label
				MOVS	r7,#0				;keep an index for iterations at r7
				LDR		r6,[r4]				;load value at r4 to r6
				B		iterate_mal			;if not at the end of table, check a new line at the loop
\end{lstlisting}

\item Malloc Error: Malloc returns an error if there is no empty space in the allocation table.
\begin{lstlisting}[caption=Malloc error label, style=customasm]
mal_error		MOVS	r0,#0				;set r0 as allocation error flag
				BX		LR					;return to Insert
\end{lstlisting}

\item Found label: If a free space in the allocation table is found, the Malloc returns its corresponding memory area to Insert. It also sets the found bit in the allocation table to 1.

\begin{lstlisting}[caption=Malloc found label, style=customasm]
found			POP		{r7}				;if found, pop r7.
				MOVS 	r0,r1				;load found node address to r0
				MOVS	r3,#1				;set r3 as 1
				LSLS	r3,r7				;shift r3 left by amount of iterations in loop
				LDR		r6,[r4]				;load value at r4 to r6
				ORRS	r6,r3,r6			;set the used bit to 1
				STR		r6,[r4]				;store new allocation table value
				BX	 	LR					;return to Insert
;//-------- <<< USER CODE END System Tick Handler >>> ------------------------				
				ENDFUNC
\end{lstlisting}
\end{itemize}
\subsection{Remove(value)}
\par This function is used to find and remove the node from the linked list. It searches for the node and removes it from the linked list. If there is not any node which has the same data with the argument this function returns the error code "4" or linked list is empty it returns the error code "3" otherwise it removes the node from the linked list and returns the success code "0".

\begin{itemize}
    \item In the beginning, remove function loads necessary addresses and checks if the linked is empty. If the linked list is empty function returns error code "3".
    \begin{lstlisting}[caption=Beginning, style=customasm]
Remove			FUNCTION			
;//-------- <<< USER CODE BEGIN Remove Function >>> ----------------------															
				PUSH	{LR}			;Preserves the address of the Sys_handler
				LDR 	R1,=DATA_MEM	;Load the address of Data memory in R1
				LDR 	R2,=FIRST_ELEMENT	;Load the address of the First_Element in R2
				LDR 	R3,[R2]		;Load the address of the first element of linked list
				CMP		R3, #0		;Compare R3 with 0
				BEQ		empty		;If linked list is empty returns error code 3
				.
				.
				.
empty			MOVS	R0, #3	;If linked list is empty returns error code 3	
				POP		{PC}		;Back to Sys_Handler
;//-------- <<< USER CODE END Remove Function >>> ------------------------				
				ENDFUNC
    \end{lstlisting}
    
    \item In the iteration part, it searches for the node to delete. If there is not any node with the given argument it returns error code "4". If it finds the node to remove branches to delete part.
    \begin{lstlisting}[caption=Iteration, style=customasm]
				MOVS 	R5, #0		;counter for loop (i)
iteration		LDR		R4, [R3]		;Load the data of the first element to R4
				CMP		R4, R0			;Compares R4 with input (R0)
				BEQ		delete		;If R4 is what user looks for program goes for delete
				MOVS 	R6, R3		;Copy the current element address to R6
				LDR		R3, [R3,#4]	;Assigns the address of the next element to R3
				CMP		R3, #0		;Looks if Array is finished yet or not
				BEQ		not_found	;If yes, program goes for not_found
				ADDS	R5, #4		;increasement of i
				B		iteration	;branching loop
				.
				.
				.
not_found		MOVS	R0, #4	;If linked list is empty returns error code 4	
				POP		{PC}		;Back to Sys_Handler

    \end{lstlisting}
    
    \item In the delete part, it first controls if the node is the first node and branches to delete first part or continues if the node is not the first element of the linked list. To delete an element it first loads the address of the next node to the previous node than removes the element from the linked list. After deleting the node it calls the Free function with the address of the node that is deleted than returns success code "0".
    \begin{lstlisting}[caption=Delete, style=customasm]
delete			CMP 	R5, #0				;compare r5 with 0, program removes the asked element
				BEQ		delete_first		;if element asked for removal is the first one, goes for delete_first	
				
				LDR		R4, [R3, #4]		;Load address of the next of the removal element to R4
				STR 	R4, [R6, #4]		;Store R4 in previous element of linked list 
				MOVS	R4, #0			;Assings 0 to R4
				STR		R4, [R3]		;Assigns 0 to data of removal element, which means deletion
				STR 	R4, [R3,#4]		;Assigns 0 to address of removal element, which means deletion
				MOVS	R0, R3			;Copy R5(index of removal element) to R0
				BL		Free			;Branch Free
				MOVS	R0, #0		;success code
				POP		{PC}		;Back to Sys_Handler
    \end{lstlisting}
    
    \item In the delete first part, it first loads the address of the next element to the FIRST ELEMENT. Then removes the element from the linked list, calls the Free function and returns success code "0".
    \begin{lstlisting}[caption=Delete\_first, style=customasm]
delete_first	LDR		R4, [R3,#4]		;Load address of the next element to R4
				STR		R4, [R2]		;Change the address of the FIRST_ELEMENT
				
				MOVS	R4, #0			;Assings 0 to R4
				STR		R4, [R3]		;Assigns 0 to data of removal element, which means deletion
				STR 	R4, [R3,#4]		;Assigns 0 to address of removal element, which means deletion
				MOVS	R0, R3			;Copy R5(index of removal element) to R0
				BL		Free		;Branch Free
				MOVS	R0, #0		;success code
				POP		{PC}		;Back to Sys_Handler
    \end{lstlisting}
\end{itemize}
\newpage
\subsection{Free(address)}
\par This function is used to clear the corresponding bit from the allocation table using the address value.

\begin{itemize}
    \item In the beginning, it loads the address of the end of the allocation table to the register 1 and address of the DATA MEM to register 2 (R2) and finds a value that we call index using the following formula.
    \begin{align*}
        \text{index} = \frac{\text{address} - R2}{8}
    \end{align*}
    \par This value(index) gives us the address of the bit from the allocation table that will be set to 0.
    \begin{lstlisting}[caption=Beginning, style=customasm]
Free			FUNCTION			
;//-------- <<< USER CODE BEGIN Free Function >>> ----------------------				
                LDR        R1, =(__AT_END-4)    ;load end address of allocation table to R1
                LDR        R2, =DATA_MEM        ;load address of data memory to R2
                SUBS    R0, R0, R2            ;substract data memory address from remova element's address
                LSRS    R0, R0, #3            ;divide R0 by 8
    \end{lstlisting}
    
    \item In the dec loop part, it compares index value with 32(4 bytes = 1 word), decrements it by 32 if bigger or equal and decrements the value at register 2 by 4(1 word) and continues until index value is smaller than 32.
    \begin{lstlisting}[caption=dec\_loop, style=customasm]
                MOVS    R3, #32			;Assign 32 to R3
dec_loop        CMP        R0, R3		;Compare R0 with R3 (32)
                BLT        dec_end		;Branch to end loop if smaller
                SUBS    R1, R1, #4		;Substract R1 by 4
                SUBS    R0, R0, R3		;Substract R0 by 32
                B        dec_loop		;Branch to loop again
    \end{lstlisting}
    
    \item After dec loop, loads 0xFFFFFFFE to register 4 that will be used to set the corresponding bit 0 using and operation.
    \begin{lstlisting}[caption= After dec\_loop, style=customasm]
dec_end            LDR        R4, =0xFFFFFFFE	;Load 111...1110 (32 bits) to R4
    \end{lstlisting}
    
    \item In the shift loop part, function compares index value with 0, decrements it by 1 if bigger, shifts the value in register 4 one bit to the left and since left shift makes the last bit 0, sets the last bit 1 again.
    \begin{lstlisting}[caption=shift\_loop, style=customasm]
                MOVS    R3, #1			;Assign 1 to R3
shift_loop      CMP        R0, #0		;Compare R0 with 0
                BEQ        shift_end		;Branch to end loop if equal
                LSLS    R4, R4, #1		;Shift left R4 one bits
                ORRS    R4, R4, R3		;Or operation R4 with R3 to set last bit 1 again, left shift makes last bit 0
                SUBS    R0, R0, #1		;Substract R0 by 1
                B        shift_loop		;Branch to loop again
    \end{lstlisting}
    
    \item After the shift loop part, the corresponding bit from the allocation table is set to 0 using the value at register 4 and "AND" operation.
    \begin{lstlisting}[caption=End of Free, style=customasm]
shift_end        LDR        R5, [R1]		;Load value inside R1(allocation table) to R5
                ANDS    R5, R5, R4		;And operation to change value with new one
                STR        R5, [R1]		;Store new value in R1(allocation table)
                BX        LR            ;Branching back to Remove function					
;//-------- <<< USER CODE END Free Function >>> ------------------------				
				ENDFUNC
    \end{lstlisting}
\end{itemize}

\subsection{LinkedList2Arr()}

In the $LinkedList2Arr$ function, The linked list that is stored in the DATA\_MEM area is copied into the ARRAY\_MEM area in order. It does not take any parameter. The function returns the success code if the linked list is properly transformed to the array. However, it returns error code 5 if linked list is empty. As shown in Listing ??, firstly the content of the ARRAY\_MEM area is cleared. In order to assign 0 to each word of ARRAY\_MEM area, clearll2a loop is used.

\begin{lstlisting}[caption=Clearing the ARRAY\_MEM area, style=customasm]
LinkedList2Arr	FUNCTION			
;//-------- <<< USER CODE BEGIN Linked List To Array >>> ----------------------			
				LDR 	R7, =ARRAY_MEM		;loads start address of array to r7
				LDR 	R5, =ARRAY_SIZE		;loads size of array to r5
				MOVS	R6, #0				;loads 0 to r6
clearll2a		SUBS	R5, R5, #4			;clears from last element of array to first element
				STR		R6, [R7,R5]			;assigns 0
				CMP		R5, R6				;compares 0 and R5
				BNE		clearll2a			;if the first element is not reached, branches to clearll2a
\end{lstlisting}

Then, Starting from the first element of the linked list, the value of each element is copied to the array. At the begining of this part of code, it checks that the linked list is empty. If FIRST\_ELEMENT is 0, it is empty and the function returns error code 5 with register 0. Otherwise, the values of linked list are transferred to the array. The next element is reached by using the address value that is stored in the second word of the element. After transformation, the function returns the success code with register 0.

\newpage
\begin{lstlisting}[caption=Transforming the linked list to the array, style=customasm]
				LDR 	R3,=FIRST_ELEMENT	;loads the address of the FIRST_ELEMENT to R3
				LDR		R4, [R3]			;loads the value of first element to R4
				CMP		R4, #0				;compare R5 with 0
				BEQ		emptyll2a			;If linked list is empty, branch to emptyll2a
				LDR		R5, [R4]			;loads the value of first element to R5
				MOVS	R6, #0				;assisgns 0 to R6 to reach the first element of the array
loopll2a		STR		R5, [R7, R6]		;assigns the value of linked list to the array
				ADDS	R6, #4				;increases inex of the array (R6) by 4
				ADDS	R4, #4				;to reach the address value of linked list element
				LDR		R4, [R4]			;assigns the address value of linked list element to R3
				LDR		R5, [R4]			;assigns the value of the next element to R5
				CMP		R4, #0				;if the address value of linked list element is not empty
				BNE		loopll2a			;branch to loop
				MOVS	R0, #0				;success code
				BX		LR					;Back to Sys_Handler
				
emptyll2a		MOVS	R0, #5				;if linked list is empty returns error code 5	
				BX		LR				;Back to Sys_Handler
;//-------- <<< USER CODE END Linked List To Array >>> ------------------------				
				ENDFUNC
\end{lstlisting}
\newpage
\subsection{WriteErrorLog(Index, ErrorCode, Operation, Data)}

In the $WriteErrorLog$ function, informations about operation that is performed in SysTick\_Handler function are written in LOG\_MEM area. The function takes Index, ErrorCode, Operation, and Data variables as parameters and 3 word is written in the area. The first 16 bits, 8 bits and other 8 bits of the first word belong to index, error code and operation. The second word belongs to the input data and the working time is stored in the third word. In order to get the working time of the System Tick Timer (in us), GetNow function is called. 

\begin{lstlisting}[caption=WriteErrorLog Function, style=customasm]
WriteErrorLog	FUNCTION			
;//-------- <<< USER CODE BEGIN Write Error Log >>> ----------------------		
				PUSH	{LR}					;pushes LR to stack
				
				LSLS	R4, R1, #8				;shifts left error code 8 times
				ADDS	R2, R2, R4				;sums operation value and shifted error code
				LSLS	R4, R0, #16				;shifts left index 16 times
				ADDS	R2, R2, R4				;sums shifted index, shifted error code and operation value
				LDR		R4, =INDEX_ERROR_LOG	;loads the address of the INDEX_ERROR_LOG to R4
				LDR		R5, [R4]				;loads the value of INDEX_ERROR_LOG to R5
				MOVS	R0, #12					
				MULS	R0, R5, R0
				LDR		R6, =LOG_MEM			;loads start address of error log array to R6
				STR		R2, [R6, R0]			;loads the first word to the array
				ADDS	R0, R0, #4				;increases index value by 4
				STR		R3, [R6, R0]			;loads the second word to the array (data)
				ADDS	R0, R0, #4				;increases index value by 4
				MOVS	R1, R0					;loads R0 value to R1
				BL		GetNow					;branch to GetNow function
				STR		R0, [R6, R1]			;loads the third word (timestamp)
				ADDS	R5, #1					;increases R5 value by 1
				STR		R5, [R4]				;loads R5 to INDEX_ERROR_LOG
				
				POP		{PC}					;returns to SysTick_Handler function
;//-------- <<< USER CODE END Write Error Log >>> ------------------------				
				ENDFUNC
\end{lstlisting}

\subsection{GetNow()}

In the $GetNow$ function, the working time of the System Tick Timer (in us) is computed by using Equations (2,3). As stated in the assignment, period of the System Tick Timer interrupt is 964 us and CPU clock frequency is 16 MHz. Reload value is 15423. SysTick\_Handler function is called every 964 us. The number of times the function was called is kept in the TICK\_COUNT. The result of the product of two values is the time until calling this SysTick\_Handler function. The result of the product of two values is the time until calling this SysTick\_Handler function. In order to calculate the elapsed time in this SysTick\_Handler function, the current value is subtracted from the reload value and the result is divided by 16. Finally, two values are added. 

\begin{equation}
    TICK\_COUNT \times Period~Of~the~System~Tick~Timer~Interrupt
\end{equation}

\begin{equation}
    (Reload~Value - System~Tick~Timer~Current~Value) / Clock~Frequency
\end{equation}
\newpage

\begin{lstlisting}[caption=GetNow Function, style=customasm]
GetNow			FUNCTION			
;//-------- <<< USER CODE BEGIN Get Now >>> ----------------------				
				LDR		R2, =TICK_COUNT			;loads the address of TICK_COUNT to R2
				LDR		R2, [R2]				;loads the value of TICK_COUNT to R2
				ADDS	R2, R2, #1				;increases TICK_COUNT value by one
				LDR		R0, =964				;assigns the period of the System Tick Timer Interrupt to R0
				MULS	R0, R2, R0				;multiplies TICK_COUNT value and 964 microseconds
				LDR  	R3, =0xE000E018			;loads the address of SysTick Current Value Register
				LDR		R3, [R3]				;loads the value of SysTick Current Value Register to R3
				LDR	 	R7, =15423				;assigns computed reload value (15423) to R7
				SUBS	R3, R7, R3				;subtracts the current value from the reload value
				LSRS	R7, R3, #4				;shifts right the result 4 times to divide by 16
				ADDS	R0, R0, R7				;calculates the final the working time
				
				BX		LR						;returns to WriteErrorLog function
;//-------- <<< USER CODE END Get Now >>> ------------------------
				ENDFUNC
\end{lstlisting}
\newpage
\subsection{SysTick\_Stop()}
\par This function stops the System Tick Timer, clears the interrupt flag of it and updates the program status.
\begin{lstlisting}[caption=SysTick\_Stop, style=customasm]
SysTick_Stop	FUNCTION			
;//-------- <<< USER CODE BEGIN System Tick Timer Stop >>> ----------------------	
				LDR	 R0, =0xE000E010			;loads the address of SysTick Control and Status Register
				MOVS R1, #0						;assigns 0 to R1
				STR	 R1, [R0]					;assigns 0 to CLKSOURCE, TICKINT and ENABLE
				
				LDR	 R0, =0xE000E014			;loads the address of SysTick Reload Value Register
				MOVS R1, #0						;assigns 0 to R1
				STR	 R1, [R0]					;assigns 0 to SysTick Reload Value Register
				
				LDR	 R0, =PROGRAM_STATUS		;loads the address of PROGRAM_STATUS global value
				MOVS R1, #2						;assigns 2 to R1
				STR	 R1, [R0]					;assigns 2 to PROGRAM_STATUS (All data operation finished)
				
				POP	 {PC}						;returns to where the SysTick_Handler function was called
;//-------- <<< USER CODE END System Tick Timer Stop >>> ------------------------				
				ENDFUNC
\end{lstlisting}
\newpage
\section{RESULTS}
The results for this experiment were as they were intended. In this section we will dive deeply into many different cases and have a look at their outcomes to analyze and critique our work.
\newline We will check the state of the program using the test data set provided for us by our teaching assistants. 
\begin{itemize}
    \item Results after initial insertions: First few inputs are all different cases of "Insert" operations into the list. We will have a look at Data Memory, Allocation Table, Array Memory and Log Memory values.
    
    \newline Looking at the Data Memory (Figure \ref{fig:data_mem}) at this point of the program, we can see that all of the insert operations have been done and pointers/data values are indeed correct.
    \begin{figure}[H]
    \centering
    \includegraphics[scale=1.5]{data_mem.png}
    \caption{Data Memory at the middle of the program}
    \label{fig:data_mem}
    \end{figure}
    
    
    
    After looking at the allocation table values (Figure \ref{fig:at_mem})we can see that the necessary bits have been set to 1. These bits are some of the very last bits on the table and this result is right considering our design for the allocation table.
    \begin{figure}[H]
    \centering
    \includegraphics[scale=1.5]{at_mem.png}
    \caption{Allocation Table at the middle of the program}
    \label{fig:at_mem}
    \end{figure}


    Log Memory will be discussed at the end of program, since early values don't change throughout the execution time of the program.
    
    Observing the Array Memory would be quite pointless at this point of the program since Linked List To Array function has not been called yet.
    
    \item Results at the end of program: We will now observe and interpret the result our code produces at the end of the program execution. 

    \newline Looking at the Data Memory (Figure \ref{fig:data_mem_son}) at this point of the program, we can see that all of the insert operations have been done and pointers/data values are indeed correct.




    \begin{figure}[H]
    \centering
    \includegraphics[scale=1.5]{data_mem_son.png}
    \caption{Data Memory after the program}
    \label{fig:data_mem_son}
    \end{figure}

    \newline After looking at the allocation table values (Figure \ref{fig:at_mem_son})we can see that the necessary bits have been set to 1 and some bits have been set back to zero. This result is right considering our design for the allocation table and the input values.

    \begin{figure}[H]
    \centering
    \includegraphics[scale=1.4]{at_mem_son.png}
    \caption{Allocation Table after the program}
    \label{fig:at_mem_son}
    \end{figure}
    
    \newline Looking at the Log Memory, we can clearly see that operation flags, error codes, tick count, input data and system time values have been stored successfully. 
    \begin{figure}[H]
    \centering
    \includegraphics[scale=1.3]{log_mem_son.png}
    \caption{Log Memory after the program}
    \label{fig:log_mem_son}
    \end{figure}
    
    \newline Looking at the (Figure \ref{fig:array_mem_son})we can see that the data that was once inside the linked list is now word by word located in the Array Memory.
    \begin{figure}[H]
    \centering
    \includegraphics[scale=1.5]{array_mem_son.png}
    \caption{Array Memory after the program}
    \label{fig:array_mem_son}
    \end{figure}

\end{itemize}
\newpage
\section{DISCUSSION}
Our noble team was able to produce the desired results without running into many issues. This can be attributed to our superb task management and teamwork. We divided the project into smaller tasks and planned the whole operation that way, which help us considerably throughout the project we must add. The main parts were: Configuring the systick and its related functions, inserting data to the list, removing data from the list and printing results. We implemented each of these parts in succession and handled their errors/log entries as we kept implementing them one by one. This workflow resulted in a smooth and favourable journey for our team. Each member has taken part in designing each of these different tasks. Hence, we were able to produce different ideas coming from different points of view. This situation was extremely fruitful when coming up with a design since it widened our options massively.  
\newline  The results have been analysed in detail in part 3 of the report, but if we have to discuss them shortly here we can say that they were on par with our expectations. We were able to get the expected outcomes for different input datasets we created which is an end result that was quite satisfying four our team.

\section{CONCLUSION}
Our noble team strongly feels that, this was a quite eye opening implementation. Until this experiment, while we were writing code in c++ and other languages we were able to use the malloc() function by typing just one little keyword and we didn't really pay much attention to what was happening behind the scenes. But after doing this experiment, we acquired some inside information about these functions and understood what is happening inside them thoroughly. This newly gained experience will surely help not only our assembly implementations but will also improve our C/C++ coding abilities as well.

\newpage
\addcontentsline{toc}{section}{\numberline {}REFERENCES}

\bibliographystyle{unsrt}
\bibliography{reference}
\newline https://developer.arm.com/documentation/dui0497/a/introduction/about-the-cortex-m0-processor-and-core-peripherals
\newline https://www.keil.com/support/man/docs/armasm/
\newline https://developer.arm.com/documentation/ddi0484/latest/
\newline Assignment PDF found in NINOVA
\end{document}

